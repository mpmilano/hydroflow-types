
\documentclass[acmsmall,review,screen,anonymous]{acmart}

%\AtBeginDocument{%
%  \providecommand\BibTeX{{%
%    Bib\TeX}}}

\setcopyright{acmcopyright}
\copyrightyear{2023}
\acmYear{2023}
\acmDOI{XXXXXXX.XXXXXXX}

\acmConference[PLDI 23]{PLDI}{June 17--21,
  2023}{Orlando, FL}

\acmPrice{15.00}
\acmISBN{978-1-4503-XXXX-X/18/06}

%%\acmSubmissionID{123-A56-BU3}


\citestyle{acmauthoryear}




\usepackage{utf8math,ttquot,mathpartir,amsmath, amssymb, hydrocomments, mathtools,multicol,xspace,stmaryrd }


\begin{document}

\title{HydroFlow: A New Dataflow Programming Language}

\author{Mae Milano}
%\authornote{Both authors contributed equally to this research.}
\email{mpmilano@berkeley.edu}
\orcid{0000-0003-3126-7771}

\author{Mingwei Samuel}
%\authornotemark[1]
\email{mingwei@berkeley.edu}
%\orcid{0000-0003-3126-7771}

\author{Shadaj Laddad}
%\authornotemark[1]
\email{shadaj@berkeley.edu}
%\orcid{0000-0003-3126-7771}

\author{Joe Hellerstein}
%\authornotemark[1]
\email{hellerstein@berkeley.edu}
%\orcid{0000-0003-3126-7771}

\affiliation{%
  \institution{UC Berkeley}
  \streetaddress{somewhere}
  \city{Berkeley}
  \state{CA}
  \country{USA}
  \postcode{94720}
}


%%
%% By default, the full list of authors will be used in the page
%% headers. Often, this list is too long, and will overlap
%% other information printed in the page headers. This command allows
%% the author to define a more concise list
%% of authors' names for this purpose.
\renewcommand{\shortauthors}{Trovato et al.}



%%
%% The abstract is a short summary of the work to be presented in the
%% article.
\begin{abstract}
  A clear and well-documented \LaTeX\ document is presented as an
  article formatted for publication by ACM in a conference proceedings
  or journal publication. Based on the ``acmart'' document class, this
  article presents and explains many of the common variations, as well
  as many of the formatting elements an author may use in the
  preparation of the documentation of their work.
\end{abstract}

%% A "teaser" image appears between the author and affiliation
%% information and the body of the document, and typically spans the
%% page.
%\begin{teaserfigure}
%  \includegraphics[width=\textwidth]{sampleteaser}
%  \caption{Seattle Mariners at Spring Training, 2010.}
%  \Description{Enjoying the baseball game from the third-base
%  seats. Ichiro Suzuki preparing to bat.}
%  \label{fig:teaser}
%\end{teaserfigure}

%\received{20 February 2007}
%\received[revised]{12 March 2009}
%\received[accepted]{5 June 2009}

%%
%% This command processes the author and affiliation and title
%% information and builds the first part of the formatted document.
\maketitle

\section{Introduction}
ACM's consolidated article template \cite{mixt}, 

\bibliographystyle{ACM-Reference-Format}
\citestyle{acmauthoryear}
\bibliography{thesis,pm-master}

\end{document}
\endinput
%%
%% End of file `sample-acmsmall-conf.tex'.
