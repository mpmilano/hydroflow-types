
\documentclass{article}
\usepackage{utf8math,ttquot,mathpartir,amsmath, amssymb, hydrocomments, mathtools,multicol,xspace}
\usepackage[margin=1in]{geometry}
\begin{document}

\section{syntax}

\newcommand{\closedprogram}{\textit{closed-program}\xspace}
\newcommand{\compiledcomponent}{\textit{compiled-component}\xspace}
\newcommand{\incast}{\textit{incast}\xspace}
\newcommand{\outcast}{\textit{outcast}\xspace}
\newcommand{\seqstart}{\textit{seq-start}\xspace}
\newcommand{\seqend}{\textit{seq-end}\xspace}
\newcommand{\chain}{\textit{chain}\xspace}
\newcommand{\op}{\textit{op}\xspace}
\newcommand{\opt}{τ_{\textit{op}}\xspace}
\newcommand{\N}{ℕ}
\newcommand{\fresh}{\textit{fresh}\xspace}
\newcommand{\inputs}{\textit{inputs}\xspace}
\newcommand{\outputs}{\textit{outputs}\xspace}

\begin{figure}
  \begin{align*}
    \closedprogram &:= "declare"~ A…~"in"~ p\\
    A &∈ \textit{handoff-names}\\
    p &:= [p]\op[p] ∣ A ∣ (p|p)
  \end{align*}
  \label{fig:syntax}
  \caption{Syntax}
\end{figure}


In figure \ref{fig:syntax}

\section{types}

\begin{figure}
  \begin{mathpar}

    {\inferrule{A : in(dₐ : τₐ), A : out(dₐ | τₐ),… ;. : ⊤ ⊢ p : τ}{⊢ "declare"~A…~"in"~p : τ}}
    
    {
      \inferrule{\inputs(\op) = τ' \\ \outputs(\op) = τ_o \\
        C;. : ⊤ ⊢ p : τ' \\ C;d.p.\op[0] :^r τ_o ⊢ p' : τ_o'}
                {{C,C'};{d : τ} ⊢ [p]\op[p'] : τ_o'}}
    
    {\inferrule[relevant-parallel]{C₁;d[n] : τ ⊢ p₁ : τ₁ \\ C₂;d[n+1] :^r τ ⊢ p₂ : τ₂}{C₁,C₂;d[n] :^r τ ⊢ (p₁ ∣ p₂) : (τ₁ ∣ τ₂)}}

    {\inferrule[substructural-fallback]{C;d : τ' ⊢ p : τ}{C;d : τ' ⊢^r p : τ}}

    {\inferrule[general-parallel]{C₁;. : ⊤ ⊢ p₁ : τ₁ \\ C₂;. : ⊤ ⊢ p₂ : τ₂}{C₁,C₂; . : ⊤ ⊢ (p₁ ∣ p₂) : (τ₁ ∣ τ₂)}}

    {\inferrule[exchange]{A : (dₐ : τₐ),  B : (d_b | τ_b) ; d : τ ⊢ p : τ}{B : (d_b | τ_b), A : (dₐ : τₐ) ⊢ p : τ}}
      
    {\inferrule[unfold]{A:d',C;d[A/d] : τ ⊢ p}{A:d',C;d : τ ⊢ p}}

    {\inferrule[fold]{A:d',C;d : τ ⊢ p}{A:d',C;d[A/d] : τ ⊢ p}}

    {\inferrule[varref-out]{}{A:out(d : τ);d : τ ⊢ A : τ}}

    {\inferrule[varref-in]{}{A:in(d : τ);. : \_ ⊢ A : τ}}
      
    \end{mathpar}
    %\includegraphics{freehand-types}
  \label{fig:types}
  \caption{Types}
\end{figure}

We're not bothering with the oaklandish split right now, it's too silly.  Open question if we need the relevance version of the rule, or can remove the $:^r$ annotation and observe that the $[n]$ version is only ever used with empty contexts for inputs to operators. Also worth wondering if we're going to be in trouble overloading the [] syntax here; time will tell. Types appear in figure \ref{fig:types}


\end{document}
